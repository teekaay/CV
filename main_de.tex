\documentclass[]{deedy-resume-openfont}
\begin{document}
%\lastupdated

\namesection{Thomas}{Klinger}{ \urlstyle{same}\url{http://github.com/teekaay} \\
\href{mailto:2klinger@informatik.uni-hamburg.de}{2klinger@informatik.uni-hamburg.de} | 040-87605114
}

%%%%%%%%%%%%%%%%%%%%%%%%%%%%%%%%%%%%%%
%
%     COLUMN ONE
%
%%%%%%%%%%%%%%%%%%%%%%%%%%%%%%%%%%%%%%

\begin{minipage}[t]{0.33\textwidth}

%%%%%%%%%%%%%%%%%%%%%%%%%%%%%%%%%%%%%%
%     EDUCATION
%%%%%%%%%%%%%%%%%%%%%%%%%%%%%%%%%%%%%%

\section{Ausbildung}

\subsection{Universiät Hamburg}
\descript{MSc Informatik}
\location{Voraussichtlich Juli 2018}
\sectionsep

\descript{BSc Informatik}
\location{Oktober 2016}
% Courses
\location{Abschlussnote: 2.5}
\sectionsep

\subsection{Vincent-Lübeck-Gymnasium Stade}
\location{Abschluss: Juni 2011 | Stade}
\sectionsep

%%%%%%%%%%%%%%%%%%%%%%%%%%%%%%%%%%%%%%
%     LINKS
%%%%%%%%%%%%%%%%%%%%%%%%%%%%%%%%%%%%%%

\section{Links}
Github:// \href{https://github.com/teekaay}
{\custombold{teekaay}} \\
Twitter://  \href{https://twitter.com/tee_okay}{\custombold{@tee\_okay}} \\
\sectionsep

%%%%%%%%%%%%%%%%%%%%%%%%%%%%%%%%%%%%%%
%     COURSEWORK
%%%%%%%%%%%%%%%%%%%%%%%%%%%%%%%%%%%%%%

\section{Besuchte Kurse}
\subsection{Master}
Neural Networks \\
Requirements Engineering \\
Entwicklung von verteilten, kontextbasierten System (Masterprojekt)
\sectionsep

\subsection{Bachelor}
Data Mining \\
Machine Learning \\
Stochastik I \& II

%%%%%%%%%%%%%%%%%%%%%%%%%%%%%%%%%%%%%%
%     SKILLS
%%%%%%%%%%%%%%%%%%%%%%%%%%%%%%%%%%%%%%

\section{Fähigkeiten}
\subsection{Programmiersprachen}
Bash \textbullet{} R \textbullet{} Java (Spring) \textbullet{} JavaScript (React, ES6) \textbullet{} Ruby
\sectionsep

\subsection{Technologie}
Linux \textbullet{} Jenkins \textbullet{} Docker \textbullet{} Apache Mesos \textbullet{} Kibana

%%%%%%%%%%%%%%%%%%%%%%%%%%%%%%%%%%%%%%
%
%     COLUMN TWO
%
%%%%%%%%%%%%%%%%%%%%%%%%%%%%%%%%%%%%%%

\end{minipage}
\hfill
\begin{minipage}[t]{0.66\textwidth}

%%%%%%%%%%%%%%%%%%%%%%%%%%%%%%%%%%%%%%
%     EXPERIENCE
%%%%%%%%%%%%%%%%%%%%%%%%%%%%%%%%%%%%%%

\section{Erfahrung}

\runsubsection{OTTO GmbH}
\descript{Software Entwickler | Werkstudent }
\location{Mai 2016 - | Hamburg, GER}
\vspace{\topsep} % Hacky fix for awkward extra vertical space
\begin{tightemize}
  \item Implementierung von Robustheits- und Resilienztests
  \item Dashboarding mit Grafana
  \item Entwicklung einer Client-Bibliothek zur Kommunikation mit Graphite in Java
  \item Datenanalyse und Visualisierung
\end{tightemize}
\sectionsep

\runsubsection{Universität Hamburg}
\descript{Masterprojekt Entwicklung verteilter, kontextbasierter Systeme | Entwickler }
\location{Oktober 2016 - Januar 2017 | Hamburg, GER}
\vspace{\topsep} % Hacky fix for awkward extra vertical space
\begin{tightemize}
  \item Implementierung einer Microservice Architektur zur Sammlung von Forschungsdaten
  im Forstumfeld
  \item Architektonische Planung
  \item Umsetzung von Continious Integration Prozessen
\end{tightemize}
\sectionsep

\runsubsection{Informatik Bibliothek}
\descript{| Studentische Hilfskraft }
\location{April 2013 – February 2016 | Hamburg, GER}
\vspace{\topsep} % Hacky fix for awkward extra vertical space
\begin{tightemize}
  \item Hilfe bei Literaturrecherche für Studenten und Wissenschaftler
  \item Wissen im Bereich wissenschaftliches Publizieren und des Verlagswesens aus bibliothekarischer Sicht gesammelt
\end{tightemize}
\sectionsep

%%%%%%%%%%%%%%%%%%%%%%%%%%%%%%%%%%%%%%
%     RESEARCH
%%%%%%%%%%%%%%%%%%%%%%%%%%%%%%%%%%%%%%

\section{Forschung}
\runsubsection{Bachelorarbeit}
\descript{}
\location{Wintersemester 2015/2016 - jetzt | Hamburg}
Titel: \textsc{Analyzing Spectral Data using Similarity Search} \\
Anwendung von Algorithmen des maschinellen Lernens auf chemische Datensätze zur Duplikatenerkennung.
\sectionsep

\runsubsection{Seminar Brain Modeling}
\descript{}
\location{Sommersemester 2015 | Hamburg}
Anwendung von Konzepten der Spieltheorie auf das Gebiet Reinforcement Learning, insbesondere in unsicheren Umgebungen. \\
Titel der Publikation: \textsc{Reward-based learning in cooperative games}. Publikation verfügbar auf ResearchGate.
\sectionsep

\end{minipage}
\end{document}  \documentclass[]{article}
